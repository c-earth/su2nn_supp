\documentclass[reprint, 10pt]{revtex4-2}

\begin{document}
\title{Symmetries in Magnetic Materials Systems and Equivariant Neural Network}

\date{\today}

\maketitle
\section{Introduction}
Representing a spin system in a material efficiently is a challenge due to the sheer number of parameters involved, 
to name a few: atomic types and their physical descriptors, atomic positions, spins, coupling constants, etc. 
Each of them also obeys different sets of rules which always make representing the system complicated especially
if one want to numerically represent it for computational purposes, e.g. machine learning.

A class of machine learning that would greatly benefit from efficient representation of the system is neural network,
which contains a large amount of training parameters and require significant amount of distinct data points with 
good (sometimes crafty) augmentations. With the advance in recent development of equivariant neural network, it provides
a great method to reduce both training parameters and augmentation for the same level of model complexity due to the
fact that the models are awared of the symmetries of the data points which limit the possible choice of outcome,
and drive the necessity of the augmentations related to said symmetries obsolete.

The goal of this work is to surve as the comprehensive mathematical guide for developing equivariant neural networks 
for spin systems.

\section{Symmetries in Spin System}
For a general spin system, one can always divided the representation of the system into the physical space information,
and the spin space information which is the internal degree of freedom of the quantum angular momentum. Hence, in addition
to the normal symmetries in E(3) for physical information, we need to consider the SU(2) properties of the spin as well.
Furthermore, we will also assume the non-relativistic treatment of the system which entails the consideration of Time
reversal symmetry as well.

\section{Symmetries and Group Representation}


\section{Irreducible Representation and Equivariant}


\section{Irreducible Representation of Spin Systems}
In order to fully utilize the symmetries of spin systems, one need to find the irreducible representations of the combined
symmetries. However, not all symmetries commute with each other which make the action of one symmetry group mix different 
irreducible representations of the other non-commuting group. This defeats the purposes of using irreducible representation 
for efficient trainging of equivariant network. Hence, in these cases, we will restrict the model to only use invariant 
information of one of the symmetries.

\section{Physical Translation ($T$)}
The physical translation of the system doesn't commute with the physical rotation which we will discuss in more
detail in the following sections. This means that the tranlation will mix the different irreducible representation of 
the rotation. Fortunately, the spin interaction directly depends on the relative distance between the involved objects
rather than their absolute position. Hence, it should be the case that we will restrict the model to only use the relative 
positions for positional inputs since it is invariant to any physical translation.

\section{Physical Permutation ($P$)}
The physical permutation of the system's objects also doesn't commute with the physical rotation. However, the quantum
mechanical laws are applied to each and every objects in the same way regardless of the label we put on each object.
Hence, it is also trivial to choose to restrict the model to be invariant of the object labeling, e.g. use sets as the
key object containers.

\section{Physical Rotation ($R$)}


\section{Physical Inversion ($I$)}


\section{Spin Rotation ($R_s$)}


\section{Time Reversal ($\Theta$)}


\section{Equivariant Neural Network of Spin System}


\section{Conclusion}


\end{document}