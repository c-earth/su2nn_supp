\documentclass[reprint, 10pt]{revtex4-2}

\begin{document}
\title{Symmetries in Spin Systems and Physical Aware Equivariant Neural Network}
\author{Abhijatmedhi Chotrattanapituk}
\affiliation{Quantum Measurement Group, Massachusetts Institute of Technology, Cambridge, MA, USA \\
            Department of Electrical Engineering and Computer Science, Massachusetts Institute of Technology, Cambridge, MA, USA}

\date{\today}

\maketitle
\section{Introduction}
Representing a spin system in a material efficiently is a challenge due to the sheer number of parameters involved, 
to name a few: atomic types and their physical descriptors, atomic positions, spins, coupling constants, etc. 
Each of them also obeys different sets of rules which always make representing the system complicated especially
if one want to numerically represent it for computational purposes, e.g. machine learning.

A class of machine learning that would greatly benefit from efficient representation of the system is neural network,
which contains a large amount of training parameters and require significant amount of distinct data points with 
good (sometimes crafty) augmentations. With the advance in recent development of equivariant neural network, it provides
a great method to reduce both training parameters and augmentation for the same level of model complexity due to the
fact that the models are awared of the symmetries of the data points which limit the possible choice of outcome,
and drive the necessity of the augmentations related to said symmetries obsolete.

The goal of this work is to surve as the comprehensive mathematical guide for developing equivariant neural networks 
for systems with known physical symmetries, particularly, in this case, spin system.

\section{Symmetries in Spin System}
For a general spin system, one can always divided the representation of the system into the physical space information,
and the spin space information which is the internal degree of freedom of the quantum angular momentum. Hence, in addition
to the normal symmetries in E(3) for physical information, we need to consider the SU(2) properties of the spin as well.
Furthermore, we will also assume the non-relativistic treatment of the system which entails the consideration of Time
reversal symmetry as well.

\section{Symmetries and Group Representation}
Symmetries, in this context, are abstraction of the actions that preserve the system. This can be extremely beneficial
since it reduces the space of physically viable solutions of the system. In order to make use of such benefit in computational
works, it is crucial to represent such action in the form that usable numerically. The most common way is to choose a 
vector space and basis to represent the system, and find the action of the symmetries as a transformation in such basis.
This is one of the hand wavy definition of representation theory which is not the focus on this works. 

\section{Irreducible Representation and Equivariance}
By definition, a function is equivariant under certain action if the effect of the action on the input deterministically
apply the corresponding action on the output. This greatly benefit computational field when we know symmetries of the 
system since there is no need to recalculate (or relearn for machine learning) everything for all other inputs generated
from applying symmetry actions on the original input if we know the corresponding actions on the output.

This benefit can be boost by the concept of irreducible representation. A system's representation is reducible if there exist
a basis of the representation such that there is a direct sum of its subspace where all group actions never mix subspaces, or
they can all be written as the direct sum of their actions on each sub space. Hence, by representing the system in irreducible
representation, it would minimize the unnecessary calculations, e.g. the part that mix subspaces. However, the reduction of
computational cost from irriducible representation comes with the increase in restriction of the actions. (outside of
the symmetry group that the irreducible representation represents) Moreover, the difficulties would also multiply when 
one consider combining multiple known symmetry groups as shown in the following sections.

\section{Irreducible Representation of Spin Systems}
In order to fully utilize the symmetries of spin systems, one need to find the irreducible representations of the combined
symmetries. However, not all symmetries commute with each other which make the action of one symmetry group mix different 
irreducible representations of the other non-commuting group. This defeats the purposes of using irreducible representation 
for efficient trainging of equivariant network. Hence, in these cases, we will restrict the model to only use invariant 
information of one of the symmetries.

\section{Physical Translation ($T$)}
The physical translation of the system doesn't commute with the physical rotation which we will discuss in more
detail in the following sections. This means that the tranlation will mix the different irreducible representation of 
the rotation. Fortunately, the spin interaction directly depends on the relative distance between the involved objects
rather than their absolute position. Hence, it should be the case that we will restrict the model to only use the relative 
positions for positional inputs since it is invariant to any physical translation.

\section{Physical Permutation ($P$)}
The physical permutation of the system's objects does commute with the physical rotation. However, the quantum
mechanical laws are applied to each and every objects in the same way regardless of the label we put on each object.
Hence, it is also trivial to choose to restrict the model to be invariant of the object labeling, e.g. use sets as the
key object containers.

\section{Physical Rotation ($R$)}
Any physical rotation of the system in 3D can be fully described by a real 3 by 3 orthogonal matrix with unit determinant
which means that the action form the SO(3) group. One popular irredicuble representation of such group is by representing
the system with spherical harmonic basis, and the actions with Wigner-D matrices. Since one of the main challenge in 
deep neural network for learning 3D objects is in the amount of rotation augmentation which grow qubically with the 
accuracy instead of linearly in the case of 2D objects, it is mostly beneficial to prioritize the model to be physical
rotation equivariant.

\section{Physical Inversion ($I$)}
The physical inversion or parity is a very important physical symmetry that determine the possibility of certain interactions.
Having this symmetry equivariance allows us to create physical awared machine learning model that can be fine tuned or separate
the interested interactions. Fortunately, physical inversion commutes with physical rotation. This means that the combined
irreducible representations are just the tensor products between their respective irredicible representations. The intuitive
irreducible representation of the physical inversion is the symmetric-antisymmetric functions with the action to be multiplication
by 1 for symmetric and by -1 for antisymmetric functions.

\section{Spin Rotation ($R_s$)}
Similar to the physical rotation, any spin rotation in spin space can be fully described by a complex 2 by 2 unitary matrix
with unit determinant which is the SU(2) group. The popular irredicible representation of this group is the spinor representation
which is representing the system with spinors, and the actions with Wigner-D matrices for spinors. We can detect the obvious
that the choice for irreducible representation for spin rotation is similar to physical rotation since SU(2) is a double cover
of SO(3) with the same Lie algebra. Hence, the irreducible representation of physical rotation is redundant with spin rotation.
Since each irreducible representation of spin rotation can be indexed by a non-negative integer or half-integer $j$
and those integer indexed representations are redundant with physical rotation, spin rotation equivariance model can
also be used as physical rotation equivariance model. One can also incoperate the physical inversion symmetry similar to the physical rotation case, 
however, the representation for the actual spin is on spin space. Therefore, in the actual system, we need to restrict
the parity of 1 to all spin representations.

\section{Time Reversal ($\Theta$)}
At first glance, the time reversal seems to be a direct copy of physical inversion, but for temporal axis, since it commutes
with all other symmetries excluding the one we already only use the invariances. However, in order for the action to preserve 
the non-relativistic Schr\"odinger equation, not only is the sign of time changed, but also the wavefunction conjugated. 
This fact will make the representation of the group action vary vastly depending on the choice of other symmetric 
representations whether the field of the representation is real or complex, and the behavior
of those bases under time reversal actions. 

For a concrete example, consider two irreducible representations of physical
rotation where both of them are representing system with spherical harmonic basis, but one use real spherical harmonic
while other use complex spherical harmonic. The first representation also use real number field which make time reversal
actions the same as physical inversion. On the other hand, the complex spherical harmonic representation require the 
representations to be of complex field which make time reversal actions change the basis (flip sign of projected angular 
momentum number, i.e. $m$), and perform conjugation on the representations. 

The situation is also different for the spin rotation representations where the real spinor bases representation can 
be of complex field. However, specifically for spin system where the spin can have interaction with the extarnal field 
and/or with other objects in the system, the choice of spinor representation with real basis functions is less desirable 
due to the potential of symmetry breaking when consider the bases as energy eigenbases.

\section{Equivariant Neural Network of Spin System}
At high level, neural network is a function of inputs and weights. To make neural network equivariance to symmetries
while also use the benefit from irreducible representation concept, every part in the neural network must also be
equivariance (some of them might need to be invariance). 

\section{Conclusion}
Knowing symmetries of the system can greatly boost the effeciency of the computations especially for the deep neural network
which can contain a huge number of parameters and every calculation reduction count. However, it is intrincically impossible 
to create general neural network that is equivariant to every symmetry due to the commutation relations between them. 
Hence, it is crucial to carefully analyze the interested system and select the set of commuted symmetries where the 
model can be best benefit from the equivariance. As for the focused system in this work, spin system contains spatial 
(physical translation, permutation, rotation, and inversion), spin (spin rotation), and temporal (time reversal) symmetries.
Physical translation doesn't commute with others and physical permutation has no physical importance, and should
be handle appropriately by using only their invariant representations. Physical rotation can be chosen to fully represented by 
a subsets of spinor representation. Physical inversion and time reversal are the only discrete symmetries considered 
which both commute with spin rotation and among themselves. Hence, we can represent them with channel mechanism when perform
the message passing in neural network with a little care for system-action representation basis choices for time reversal. 

With the procedural structure of this work, one can always apply the same analytical steps to create the equivariant 
neural network that is tailored to any system with known symmetries.

\end{document}